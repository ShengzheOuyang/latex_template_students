\documentclass[ a4paper,
                oneside,
                toc=bibliography,
                toc=listof
                ]{scrbook}

\usepackage[ngerman, english]{babel}

% This class does the ISW styling for you (together with scrbook).
%
% It handles the following:
% - Proper input and font encoding (Just type, don't care about the LaTeX compiler you use or how to type German umlauts)
% - Fonts with ligatures and kerning (Tex Gyre fonts are used, part of every LaTeX installation, text is nice to read)
% - Bibliography styling for biblatex (declare your bibliography file and you are ready to go)
% - Provide command for title page (\makeISWtitle) and declaration of originality ( \declarationOfOriginality)
% - Loads packages "biblatex" and "graphics"
\usepackage[
    type=master, % bachelor, study, bachelorproject
]{iswthesis}

%Path to .bib-File(s) for BibLatex
\addbibresource{bibliography.bib}
% \addbibresource{someOtherBibFile}

\author{Max Mustermann}
\placeOfBirth{Stuttgart}
\address{Seidenstraße 36, 70174 Stuttgart}
\major{Mechatronik}
\title{How to train a hamster}
\titleTranslated{Wie man einen Hamster trainiert}
\matrnr{1234567}
\date{\today}
\supervisor{My supervisor, M.Sc.}
\professor{Prof. Dr.-Ing. Oliver Riedel}

\begin{document} 
    \frontmatter
    \makeISWtitle

    \declarationOfOriginality

    % Kurzfassung/Abstract
    
    \cleardoublepage
    \tableofcontents
    

    \mainmatter

    % ********************************************************************
    % Write your own contents here:
    % ********************************************************************

    % ********************************************************************
    % End of contents
    % ********************************************************************

    \backmatter
    \cleardoublepage
    \printbibliography

    \cleardoublepage
    \listoffigures
    \cleardoublepage
    \listoftables
    \cleardoublepage
    % Acronyms

    % Appendix, if needed:


\end{document}