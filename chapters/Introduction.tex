\chapter{Introduction}
\section{Movivation}
    A wire harness consists of electrical cables, that are used for connecting all electrical and electronic (E/E) components in the automotive vehicle, like 
    control units, sensors, and actuators\cite{TROMMNAU2019387}\cite{NGUYEN2021379}. Due to the increasing of electrification of automobile components, 
    there is also a growing importance of wire harness, such that the optimization of wire harness assembly is especially important for manufacturers,
    customer and research\cite{HEISLER2021260}. Due to the high degree of flexibility and occlusions inherent in branched deformable linear objects (BDLOs) 
    including wire harnesses, most harness assembly work is still performed by humans\cite{10284109}.The automatic manipulation of robotics can enhance efficiency 
    significantly and ease labor shortages, especially in repetitive and time-consuming processes\cite{van2012automatic}.
    But thanks to the specialized characteristics of the wire harness previously described, traditional manipulation strategies and methods used for rigid objects 
    cannot be applied to wire harnesses \cite{9952859}.
    Achieving their automatic manipulation such as fixing the wire harness on a car body or a device box\cite{JIANG201552} also poses a great challenge.  
    To perform the subsequent automation of the harness assembly, some key points on branches needs to be detected for the purpose of robotic gripping and
    the topology of the harness needs to be reconstructed using vision sensors, e.g. cameras, as well as computer vision algorithms\cite{9195380}. 
    This process could be simpler and more straightforward by utilizing an artificial intelligence approach, i.e., keypoint detection\cite{8892980}. 
    Therefore, training a deep learning model to predict the BDLOs' keypoints and reconstructing the topology become necessary.
\section{Goals of the thesis}
    The theis initially aims to adress three key aspects:
\begin{itemize}
    \item [(1)] \textbf{Develop a deep learning based AI model for extracting spline modeled segments from an image of a wire harness.}\\ 
    The primary goal is to develop a deep learning based AI model to extract the keypoints on segments of wire harness precisely.
    Interpolate the extracted key points as spline to reconstruct the segments. 
    \item [(2)] \textbf{Generate datasets and expand artificially through innovative approaches.}\\ 
    For a robust and effect training result of the deep learning model, the dataset should be big enough. At the beginning, a small dataset will be 
    generated for fast verification of the idea, which is explained above. Once the idea is verified, the dataset should be expanded artificially.
    \item [(3)] \textbf{Evaluate the generated AI model and compare with other existed methods.} \\ 
    Eventually, a series of experiments need to be done to prove the feasibility and end effects of the method. Comparative experiments with other existed methods  
    are also needed to prove the advantages of the method, as well as what improvements are possible in the future.
\end{itemize}
\section{Thesis Outline}
This section is intended to present the structure of the thesis. The thesis consists of three main parts:
\begin{itemize}
    \item [(1)] \textbf{Introduction and the state-of-the-art technologies}\\ 
    In this part, the motivation, the goal of the thesis will first be clarified. 
    Some of the fundamentals of deep learning will just be presented so that more specific problems like keypoints detection can be introduced.
    After that some state-of-the-art technologies especially for vision tasks will be introduced which are used as a theoretical basis and support for the thesis 
    as well as the future research.
    \item [(2)] \textbf{Theoretical concepts and explanation of the model}\\ 
    The concepts of wire harnesses' keypoints detection will be explained in this part. Following conception of the keypoints detection, the artificial method 
    for expanding the dataset of wire harnesses' images will be presented respectively. Some tools which are used for the thesis will also be introduced.
    After the introduction of the conceptual part, the details of the constructed model will be explained.
    \item [(3)] \textbf{Evaluate the generated AI model and compare with other existed methods.} \\ 
    In this section, many experiments will be carried out. The configurations of the experiments, and the results will be shown to intuitively present the 
    effectiveness of the model. Some problems of the model will also be optimized so that the detection of keypoints will become more accurate. 
    Finally, the model will be compared with other models that are already well known, to summarize the strengths, and the areas that need to be improved.
  \end{itemize}